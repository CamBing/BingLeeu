\documentclass[11pt,preprint, authoryear]{elsarticle}

\usepackage{lmodern}
%%%% My spacing
\usepackage{setspace}
\setstretch{1.2}
\DeclareMathSizes{12}{14}{10}{10}

% Wrap around which gives all figures included the [H] command, or places it "here". This can be tedious to code in Rmarkdown.
\usepackage{float}
\let\origfigure\figure
\let\endorigfigure\endfigure
\renewenvironment{figure}[1][2] {
    \expandafter\origfigure\expandafter[H]
} {
    \endorigfigure
}

\let\origtable\table
\let\endorigtable\endtable
\renewenvironment{table}[1][2] {
    \expandafter\origtable\expandafter[H]
} {
    \endorigtable
}


\usepackage{ifxetex,ifluatex}
\usepackage{fixltx2e} % provides \textsubscript
\ifnum 0\ifxetex 1\fi\ifluatex 1\fi=0 % if pdftex
  \usepackage[T1]{fontenc}
  \usepackage[utf8]{inputenc}
\else % if luatex or xelatex
  \ifxetex
    \usepackage{mathspec}
    \usepackage{xltxtra,xunicode}
  \else
    \usepackage{fontspec}
  \fi
  \defaultfontfeatures{Mapping=tex-text,Scale=MatchLowercase}
  \newcommand{\euro}{€}
\fi

\usepackage{amssymb, amsmath, amsthm, amsfonts}

\usepackage[round]{natbib}
\bibliographystyle{natbib}
\def\bibsection{\section*{References}} %%% Make "References" appear before bibliography
\usepackage{longtable}
\usepackage[margin=2.3cm,bottom=2cm,top=2.5cm, includefoot]{geometry}
\usepackage{fancyhdr}
\usepackage[bottom, hang, flushmargin]{footmisc}
\usepackage{graphicx}
\numberwithin{equation}{section}
\numberwithin{figure}{section}
\numberwithin{table}{section}
\setlength{\parindent}{0cm}
\setlength{\parskip}{1.3ex plus 0.5ex minus 0.3ex}
\usepackage{textcomp}
\renewcommand{\headrulewidth}{0.2pt}
\renewcommand{\footrulewidth}{0.3pt}

\usepackage{array}
\newcolumntype{x}[1]{>{\centering\arraybackslash\hspace{0pt}}p{#1}}

%%%%  Remove the "preprint submitted to" part. Don't worry about this either, it just looks better without it:
\makeatletter
\def\ps@pprintTitle{%
  \let\@oddhead\@empty
  \let\@evenhead\@empty
  \let\@oddfoot\@empty
  \let\@evenfoot\@oddfoot
}
\makeatother

 \def\tightlist{} % This allows for subbullets!

\usepackage{hyperref}
\hypersetup{breaklinks=true,
            bookmarks=true,
            colorlinks=true,
            citecolor=blue,
            urlcolor=blue,
            linkcolor=blue,
            pdfborder={0 0 0}}


% The following packages allow huxtable to work:
\usepackage{siunitx}
\usepackage{multirow}
\usepackage{hhline}
\usepackage{calc}
\usepackage{tabularx}
\usepackage{booktabs}
\usepackage{caption}
\usepackage{colortbl}

\urlstyle{same}  % don't use monospace font for urls
\setlength{\parindent}{0pt}
\setlength{\parskip}{6pt plus 2pt minus 1pt}
\setlength{\emergencystretch}{3em}  % prevent overfull lines
\setcounter{secnumdepth}{5}

%%% Use protect on footnotes to avoid problems with footnotes in titles
\let\rmarkdownfootnote\footnote%
\def\footnote{\protect\rmarkdownfootnote}
\IfFileExists{upquote.sty}{\usepackage{upquote}}{}

%%% Include extra packages specified by user
% Insert custom packages here as follows
% \usepackage{tikz}

%%% Hard setting column skips for reports - this ensures greater consistency and control over the length settings in the document.
%% page layout
%% paragraphs
\setlength{\baselineskip}{12pt plus 0pt minus 0pt}
\setlength{\parskip}{12pt plus 0pt minus 0pt}
\setlength{\parindent}{0pt plus 0pt minus 0pt}
%% floats
\setlength{\floatsep}{12pt plus 0 pt minus 0pt}
\setlength{\textfloatsep}{20pt plus 0pt minus 0pt}
\setlength{\intextsep}{14pt plus 0pt minus 0pt}
\setlength{\dbltextfloatsep}{20pt plus 0pt minus 0pt}
\setlength{\dblfloatsep}{14pt plus 0pt minus 0pt}
%% maths
\setlength{\abovedisplayskip}{12pt plus 0pt minus 0pt}
\setlength{\belowdisplayskip}{12pt plus 0pt minus 0pt}
%% lists
\setlength{\topsep}{10pt plus 0pt minus 0pt}
\setlength{\partopsep}{3pt plus 0pt minus 0pt}
\setlength{\itemsep}{5pt plus 0pt minus 0pt}
\setlength{\labelsep}{8mm plus 0mm minus 0mm}
\setlength{\parsep}{\the\parskip}
\setlength{\listparindent}{\the\parindent}
%% verbatim
\setlength{\fboxsep}{5pt plus 0pt minus 0pt}



\begin{document}

\begin{frontmatter}  %

\title{Finding the Best Rand Hedge}

\author[Add1]{Cameron Bing}
\ead{17140552@sun.ac.za}

\author[Add1,Add2]{Leeuwner Esterhuysen}
\ead{leeuwner.93@gmail.com}




\address[Add1]{Stellenbosch University, Cape Town, South Africa}
\address[Add2]{Stellenbosch University, Cape Town, South Africa}

\cortext[cor]{Corresponding author: Cameron Bing}

\begin{abstract}
\small{
This paper employs a multiple regression and DCC approach to identify
optimal hedges against volatility in the Rand exchange rate. The
approach allows for both fixed and time-varying methods, and identifies
the top 10 stocks and ETFs readily available to investors on the
Johannesburg Stock Exchange.
}
\end{abstract}

\vspace{1cm}

\begin{keyword}
\footnotesize{
DCC \sep Multiple Regression Analysis \sep Rand Hedge \\ \vspace{0.3cm}
\textit{JEL classification} L250 \sep L100
}
\end{keyword}
\vspace{0.5cm}
\end{frontmatter}



%________________________
% Header and Footers
%%%%%%%%%%%%%%%%%%%%%%%%%%%%%%%%%
\pagestyle{fancy}
\chead{}
\rhead{}
\lfoot{}
\rfoot{\footnotesize Page \thepage\\}
\lhead{}
%\rfoot{\footnotesize Page \thepage\ } % "e.g. Page 2"
\cfoot{}

%\setlength\headheight{30pt}
%%%%%%%%%%%%%%%%%%%%%%%%%%%%%%%%%
%________________________

\headsep 35pt % So that header does not go over title




\section{\texorpdfstring{Introduction
\label{Introduction}}{Introduction }}\label{introduction}

\section{\texorpdfstring{Literature Review
\label{Lit_Review}}{Literature Review }}\label{literature-review}

The concurrent volatile nature of the South African Rand has brought
about a widespread search for the best strategy aimed at protecting
capital against exchange rate volatility. This paper makes use of Baur
and Lucey (\protect\hyperlink{ref-baur2010}{2010}) `s definitions of
so-called `safe havens' and `hedges'. They define a safe haven as an
asset that is negatively related to another asset or groups of assets
during periods of high market volatility. Furthermore, they define a
hedge as an asset that that is negatively related to another asset or
groups of assets, on average. In terms of the South African equity
market, there are various equities that may potentially provide
protection against rand weakness, and hence act as a rand hedge. This is
due to the fact that a significant share of companies listed on the
Johannesburg Stock Exchange (JSE) has significant offshore exposure,
either through selling products and services that are denominated in
foreign currencies, or through significant offshore operations. As a
result, such companies will experience an increase in rand-denominated
revenue during periods where the rand depreciates. In theory, these
increases in revenue should increase the value of these companies and
consequently lead to higher share price valuations. This phenomenon
subsequently results in a positive statistical relationship between the
depreciation of the rand and the appreciation of the relevant share
price, indicating the rand hedge potential of such a share. Another
potential rand hedge strategy involves the purchasing of commodities.
Since commodities are priced in dollars, their value increases as the
rand weakens, hence serving as a hedge against the depreciation of the
rand. Prior research on this topic in South Africa is relatively
limited. Barr, Holdsworth, and Kantor
(\protect\hyperlink{ref-barr2007}{2007}) made use of a regression model
in order to investigate the relationships between the top 40 shares
listed on the JSE and the rand-dollar exchange rate. The findings of
their study imply that certain local equities can be compiled into a
given domestic portfolio that could serve as an effective and consistent
hedge against rand weakness. The same authors applied a GARCH regression
approach in 2007 to study the relation between the same two variables:
the top 40 shares listed on the JSE and the rand-dollar exchange rate.
In this study, their findings indicate significant variations in the
correlations in the correlations between the rand-dollar exchange rate
and various shares. Some shares, however, are identified as effective
hedges against rand depreciation (see Barr, Holdsworth, and Kantor
(\protect\hyperlink{ref-barr2007}{2007})). There exists a vast
international literature on the practical application of studying
co-movements between various financial returns series in an attempt to
hedge an investment portfolio against currency fluctuation. Fang and
Miller (\protect\hyperlink{ref-fang2002}{2002}) employed a bivariate
GARCH-M model in order to study the co-movements between stock market
returns and currency depreciation. Their findings suggest that some
degree of temporal dependence between the conditional variance of
currency depreciation and stock market returns. Mukherjee and Naka
(\protect\hyperlink{ref-mukherjee1995}{1995}) and Kearney
(\protect\hyperlink{ref-kearney1998}{1998}) find corroborating results,
with their respective findings suggesting a cointegrating relationship
between stock market returns and the exchange rate. The ability to
understand and predict the temporal dependence in the second-order
moments and to control for the second-order temporal persistence of
asset returns, has various financial econometric applications (Bauwens,
Laurent, and Rombouts \protect\hyperlink{ref-bauwens2006}{2006}).
Kennedy and Nourzad (\protect\hyperlink{ref-kennedy2016}{2016}) state
that increased exchange rate volatility leads to a statistically
significant, positive impact on the volatility of stock market returns
when the main sources of financial volatility are controlled for. The
findings of Baur and Lucey (\protect\hyperlink{ref-baur2010}{2010}), who
analysed the time-varying correlations between gold and a collection of
other assets in Germany, the UK and US, suggest that gold serves as a
safe haven for equities in all of these countries. Ciner, Gurdgiev, and
Lucey (\protect\hyperlink{ref-ciner2013}{2013}) employed a DCC model
with GARCH specification in order to determine the hedging ability of
multiple assets against the British pound and US dollar. Their findings
suggest that gold serves as a potential hedge against exchange rate
volatility for both of these two currencies.

\section{\texorpdfstring{Data \label{Data}}{Data }}\label{data}

\section{Methodology}\label{methodology}

This study employs two methodologies to investigate which JSE-listed
financial instruments provide the best hedge against volatility in the
Rand exchange rate. The first method utilised is a regression model
(\ref{regression}), following which a Dynamic Conditional Correlation
(DCC) model \ref{dcc} is used to investigate time-varying corellations
between various JSE-listed instruments and the Rand/US Dollar exchange
rate.

\subsection{\texorpdfstring{Regression Model
\label{regression}}{Regression Model }}\label{regression-model}

A multiple regression approach was employed to investigate the static
correlations between the Rand exchange rate and the various assets and
financial instruments covered in our data set. The initial regression
model, as specified as in equation \ref{eq1}, was run to investigate the
relationship between the assets covered in the data set and the Rand/US
Dollar exchange rate:

\begin{align} 
Return_t = \beta_0 + \beta_1 R_t + \epsilon_t \label{eq1}
\end{align}

where \(Return_t\) refers to the first difference of the log returns of
the assets and \(R_t\) to the dlog Rand/US Dollar exchange rate returns
at time \(t\). This specification includes covers all dates within the
data set.

Following these results, the data set was stratified in order to isolate
the analysis to times of high volatility, both positive and negative, in
the Rand exchange rate. This model was specified as follows:

\begin{align} 
Return_t = \beta_0 + \beta_1 R.pos.vol_t + \beta_2 R.neg.vol_t + \epsilon_t \label{eq2}
\end{align}

where \(R.pos.vol_t\) and \(R.neg.vol_t\) refer to dates where the Rand
exchange rate experienced periods of high positive and negative
volotility, repectively. The distinction between times of high and
relatively low volatility is important as this study's findings will be
most relevant to investors in times of high unstability in the Rand.
Furthermore, it allows us to minimize noise in the study which may drive
nonsensical results.

\subsection{DCC Model}\label{dcc-model}

This study utilises DCC Multivariate Generalized Autoregressive
Conditional Heteroskedasticity (MV-GARCH) to isolate the time-varying
conditional correlations between an array of JSE-listed stocks and ETFs.
This technique offers a parsimonious approach to MV colatility modelling
by relaxing the constraint of a fixed correlation structure which
imposed in other modelling techniques. The results of which allow us to
study whether fluctuations in the Rand exchange rate influence the
aforementioned financial instruments. This information can then be
reinterpreted as an indication of the best hedging options available to
investors in the South African market. In contrast to the regression
model, as described in section \ref{regression}, this method allows us
to assess the dynamic hedging potential of the assets covered in our
data set.

\section{Results}\label{results}

\section{Conclusion}\label{conclusion}

\section*{References}\label{references}
\addcontentsline{toc}{section}{References}

\hypertarget{refs}{}
\hypertarget{ref-barr2007}{}
Barr, GDI, CG Holdsworth, and BS Kantor. 2007. ``Portfolio Strategies
for Hedging Against Rand Weakness.'' \emph{South African Journal of
Accounting Research} 21 (1). Taylor \& Francis: 81--101.

\hypertarget{ref-baur2010}{}
Baur, Dirk G, and Brian M Lucey. 2010. ``Is Gold a Hedge or a Safe
Haven? An Analysis of Stocks, Bonds and Gold.'' \emph{Financial Review}
45 (2). Wiley Online Library: 217--29.

\hypertarget{ref-bauwens2006}{}
Bauwens, Luc, Sébastien Laurent, and Jeroen VK Rombouts. 2006.
``Multivariate Garch Models: A Survey.'' \emph{Journal of Applied
Econometrics} 21 (1). Wiley Online Library: 79--109.

\hypertarget{ref-ciner2013}{}
Ciner, Cetin, Constantin Gurdgiev, and Brian M Lucey. 2013. ``Hedges and
Safe Havens: An Examination of Stocks, Bonds, Gold, Oil and Exchange
Rates.'' \emph{International Review of Financial Analysis} 29. Elsevier:
202--11.

\hypertarget{ref-fang2002}{}
Fang, WenShwo, and Stephen M Miller. 2002. ``Dynamic Effects of Currency
Depreciation on Stock Market Returns During the Asian Financial
Crisis.''

\hypertarget{ref-kearney1998}{}
Kearney, Colm. 1998. ``The Causes of Volatility in a Small,
Internationally Integrated Stock Market: Ireland, July 1975--June
1994.'' \emph{Journal of Financial Research} 21 (1). Wiley Online
Library: 85--104.

\hypertarget{ref-kennedy2016}{}
Kennedy, K, and F Nourzad. 2016. ``Exchange Rate Volatility and Its
Effect on Stock Market Volatility.'' \emph{International Journal of
Human Capital in Urban Management} 1 (1). Human Resource Development
Deputy: 37--46.

\hypertarget{ref-mukherjee1995}{}
Mukherjee, Tarun K, and Atsuyuki Naka. 1995. ``Dynamic Relations Between
Macroeconomic Variables and the Japanese Stock Market: An Application of
a Vector Error Correction Model.'' \emph{Journal of Financial Research}
18 (2). Wiley Online Library: 223--37.

% Force include bibliography in my chosen format:
\newpage
\nocite{*}
\bibliography{}





\end{document}
